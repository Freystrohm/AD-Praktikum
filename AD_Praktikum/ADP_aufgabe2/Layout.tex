% das Papierformat zuerst
\documentclass
[
	a4paper, 
	11pt, 
	headings = normal,
	automark
]
{scrartcl}

% deutsche Silbentrennung und übersetzung von Standardtexten
\usepackage[ngerman]{babel}

% wegen deutschen Umlauten wichtig: Auf editor Formatierung achten und packet antsprechend anpassen!
\usepackage[utf8]{inputenc}

% Komascript für Kopf- und Fusszeile
\usepackage{scrpage2}

% Header un Footer mit scrheadings aus dem scrpage package defnieren
\pagestyle{scrheadings}
%setzen der header parameter i = inner o = outer c= center
\ihead{\Titel\\[0.5ex]\headmark}
\chead{}
\ohead{}
%setzen der Footer parameter
\ifoot{}
\cfoot{\pagemark}
\ofoot{}
%bestimmt was bei \headmark eingefügt wird
%\automark[section]{section}
%Setzen der Headline linie
\setheadsepline[text]{0.01pt} 
\setlength{\headheight}{1.1\baselineskip}

%Zum Einbinden von Bildern
\usepackage{graphicx}
\graphicspath{{Dateien/}}

\usepackage{chngcntr}
\counterwithin{figure}{section}